\chapter{Background}

This chapter provides general necessary background on mashup applications, Single Sign-On services and its Facebook flavor.  Related works that are of interest with respect to certain techniques or specific applications are given later in the respective sections.

\section{Mashup Applications}
\label{sec:bg_mashup_apps}
As we have stated in the introduction, modern applications are often composed of first party code and many third-party modules, which makes these applications a complex mashup~\cite{MashupDef}.  Mashups are common in web and mobile applications, as shown here in Figure~\ref{fig:mashup_example} as examples.

\begin{figure}[bth]
\centering
\includegraphics[width=\textwidth]{figures/chapter2/mashup_example}
\caption{Modern mashup application examples}
\label{fig:mashup_example}
\end{figure}

\subsection{Embedding mechanisms}

The left of Figure~\ref{fig:mashup_example} is a screenshot taken from Yelp.com.  This page contains a number of third-party components.  The blue rounded rectangles highlight the advertisements injected onto this page.  In a web application context, the display of such advertisement is often bootstrapped by a JavaScript snippet directly embedded in the host page.  In addition to the elements that can be visually seen in the figure, there are numerous analytics and tracking scripts from multiple parties installed on this page, as shown in Figure~\ref{fig:analytics_example}.  The green rectangle highlights the Google Maps module embedded by Yelp.  Often times widgets like Google Maps, Facebook Like button or friend status are embedded in an \emph{iframe} in the host page.  

The right of Figure~\ref{fig:mashup_example} shows a screenshot taken from a popular Android application GasBuddy.  The application is used to find cheap gas stations near the user, and it embeds an advertisement from Google Admob, shown in the blue rounded rectangle.  To embed a mobile ad, the application developer often downloads an SDK from the advertisement provider and include it into the packaged application.  The developer will then need to specify where he or she wants the ad to be displayed, and any additional information that may help to increase the conversion rate.  Similar to the web platform, mobile applications also see a fast growing in the popularity of integrating analytics and tracking code.  

\begin{figure}[bth]
\centering
\includegraphics{figures/chapter2/analytics_example}
\caption{Analytics and Tracking Scripts on Yelp.com}
\label{fig:analytics_example}
\end{figure}

\subsection{Third-Party Script Privileges}
Regardless of whether the third-party has malicious intent itself or if it is compromised by an adversary, such third-party modules pose security and privacy risks for the host and its users.  The direct inclusion of scripts basically grants the third-party \emph{full} access to the host resource.  The privilege ranges from reading sensitive page content, modifying page layout to accessing user credentials such as password typed into a login form, or \code{document.cookie} which empowers an adversary to impersonate the current user.  On the contrary, accesses from a widget that is embedded in an iframe whose origin is different from the host are subject to the same-origin policy~\cite{SOP}, and therefore are completed blocked from all host resources.

We give an example in snippet~\ref{lst:exampleScriptAccess} to show what a directly embedded malicious script can do:  

\lstinputlisting[caption={Example content of a malicious script},label={lst:exampleScriptAccess}]{listings/exampleScriptAccess.js}

After creating an image element in line 1, the script uses standard DOM APIs to obtain all first-party cookies from the host (line 2), as well as the password if the user has typed in (line 3).  It then sends this information off to evil.com by setting the source attribute of the newly created image element (line 4).  Although an asynchronous \code{XMLHTTPRequest} (i.e. AJAX request) is restricted by the same-origin policy and therefore cannot be used to exfiltrate this information, setting source to an image is not and such information can be transferred as GET parameters to the URL.  To obtain user passwords with greater success, the above code can be set to execute as an \code{onbeforeunload} event handler for the document object, which is triggered upon page navigation --- likely when the user has input his or her password and clicked the `login' button.

The panacea appears to be simple: the host should embed all third-party code in iframes with different origins.  However, online advertising these days always employs history and contextual targeting~\cite{Gill:2013:BPF:2504730.2504768}.  This means the advertisements are fetched catering to the user's browsing history and the page content.  Accesses to these resources however, will be blocked by the \emph{all-or-nothing} access model enforced by the same-origin policy in all major browsers today if they were embedded in an iframe with different origin.  Although sometimes user tracking can be done by embedding an (almost) invisible image (a.k.a. web bug), any advanced tracking and analytics functionality can only be realized by a directly embedded script.  

The high risks exposed by embedding third-party script is directly caused by the inflexibility of same-origin policy, which essentially lead to the host abandoning the built-in protection for functionality and advertisement revenue.  Motivated by this, we present our solution towards a much more flexible protection mechanism in Chapter~\ref{sec:esorics}.  To this end, we are focused on protecting web applications; similar approach to cover mobile applications is a promising future research direction but may pose additional challenges and is out of the scope of this dissertation.

\section{Single Sign-On Service}

Integrating security-critical third-party services, even when the service provider is fully trusted, may bring vulnerabilities to the application.  In this section, we describe one particular types of such services --- Single Sign-On in more details, which is necessary for readers to understand Chapter~\ref{sec:explication} and~\ref{sec:ssoscan}.

\begin{figure}[bth]
\centering
\includegraphics[width=\textwidth]{figures/chapter2/sso_example}
\caption{Single Sign-On enabled applications}
\label{fig:sso_example}
\end{figure}

Single Sign-On (SSO) service enables an relying party (RP) application developer to quickly implement its authentication functionality by redirecting users to log in through their account at the identity provider (IdP).  The application users also benefit from this since they do not need to remember and type in user names/passwords each time they log in to the application.  The relying party has the added benefit of possibly obtaining more information from the user's Facebook profile than they normally can through traditional registration, and may request privilege to post on users' walls to promote their business.  Figure~\ref{fig:sso_example} shows two screenshots from pinterest.com and huffingtonpost.com, and as long as the user is currently logged in to Facebook and has previously granted permissions to those two applications, he or she can simply click on the button to login.  

Up to date, the most widely deployed identity provider by far is Facebook, but several other popular options such as Twitter, Google Plus and Microsoft Live.  There is no official statistics from these major identity providers about their adoption rate, but the results we obtained by running SSOScan (described in Chapter~\ref{sec:ssoscan}) show that almost 10\% of the top-20,000 ranked sites implement Facebook SSO.

\subsection{Integration approach}

The most recommended approach to integrate SSO service is to use an SDK or widget provided by the IdP.  For example, Facebook issues JavaScript SDK for front-end integration of web applications, and many SDK choices for various back-end framework such as PHP and Python.  The developers will need to refer to documentations about the pre-defined functions in the SDK and call them when the user elects to login through SSO and when the SSO credentials are presented to the back-end server.  It is worth mentioning that although all the heavy-lifting (e.g. signature verification, request parameter crafting) is done by the SDK functions already, the developer would still need to make the calls to those functions to ensure a secure implementation.

If experienced developers wish to further control and customize the process, IdPs like Facebook also provide documentations to their RESTful APIs.  Developers can craft HTTP requests to such APIs and parse the responses themselves without the help of SDKs, however, such process often requires a deeper understanding of the SSO protocol.

Another growing business in this area is to delegate the integration to a ``fourth party service'', for example Janrain or Gigya.  Such service often helps the RP to integrate more than one IdPs together in a ``toolbox'', and sometimes provide additional analytics statistics about their visitors.  

Although direct client support for SSO service is much less popular than the above mentioned approaches, it has been evolving fast as of late.  Devices running the Android operating system offer built-in SSO support for Google Plus accounts.  Mozilla is also promoting its SSO service --- Mozilla Persona, which is built into the Mozilla Firefox Browser.  The advantages of using a natively-supported SSO service include SDK-free integration and push-updates to the SSO code base should vulnerabilities be discovered.

Finally, before starting to implement SSO, the application developer has to register the application at identity provider's service management center.  For Facebook, this is located at \url{http://developers.facebook.com/}.  This step is important, since Facebook needs to set up a secret known only to the developer and Facebook, and issue a valid application ID to the developer.  Such information are essential for correctly requesting and parsing in the SSO process.

\subsection{The SSO protocols}
\label{sec:bg_sso_protocol}
All major identity providers follow one of the Single Sign-On protocol standards, the most popular of which are OAuth~\cite{OAuth2.0} (version 2.0 and 1.0a) and OpenID~\cite{openID}.  For this dissertation we are going to focus on the OAuth 2.0 protocol, which is used by Facebook SSO and Microsoft Live SSO service.  Individual IdPs may extend the protocol slightly and use different terms for credential types defined in the standard, and we highlight the workflow of Facebook's implementation of OAuth 2.0 which serves readers the best for understanding the rest of the dissertation.

\subsection{Facebook OAuth 2.0 workflow}

A typical Facebook OAuth flow involves three parties, as shown in Figure~\ref{fig:oauth_workflow}.  Alice (the user) first visits a web application (1) and elects to use SSO to login.  She is then redirected (2) to the Facebook's SSO entry point with a set of pre-configured parameters.  After Facebook confirms her login credentials (3), it will issue to Alice's browser the requested OAuth credentials (4), which will then be forwarded to the application back-end server (5).  The application server confirms the identity by either checking the credential signature or sending a round-trip traffic to Facebook depending on the credential type (6), and if everything checks out, authenticates the client as Alice (7).

\begin{figure}[bth]
\centering
\includegraphics[width=\textwidth]{figures/chapter2/oauth_workflow}
\caption{Typical OAuth Workflow}
\label{fig:oauth_workflow}
\end{figure}

In these steps, the relying party developer needs to pay special attention to step (2), (5) and (6) to ensure a vulnerability-free implementation, while the Facebook's back-end server takes care of (3) and (4).

\subsection{Facebook Connect credential types}

The OAuth 2.0 standard specifies that the IdP should return different types of credentials (shown in step 4 of Figure~\ref{fig:oauth_workflow}) based on different \emph{grant\_type} values submitted in the HTTP request.  Here we list the ones Facebook adopts, including an extended, Facebook-specific type.

\begin{itemize}

\item \textbf{access\_token.} \emph{Access\_token}s is the basic OAuth credential type designed for authorization purpose.  It is a bearer token, and represents user's granted permission to the token holder.  A Facebook \emph{access\_token} is only issued when a request to the OAuth entry point API explicitly asked for it.  An API call made to request the token must contain the application ID (set up earlier) and requested permissions.  After the user agrees, an access\_token bound to those permissions is issued to the developer, which can be used as credential to call further Facebook \emph{graph} APIs and request additional user information.  \emph{Access\_token} is often valid from several hours to several days but can be extended to a much longer period by calling a token refresh API.

\item \textbf{code.} A Facebook \emph{code} credential is issued by default to calls which do not specify a \emph{grant\_type}, and can be used for both authentication and authorization purposes.  By itself, \emph{code} does not contain any meaningful information to the developer, is only valid for a short period of time, and has to be used to exchange a valid \emph{access\_token}.  To do this, the developer needs to append its \emph{client\_secret} (set up earlier) with the \emph{code} in the API call to retrieve the token.  The \emph{access\_token} obtained this way can be used to request additional user information as described above.

\item \textbf{signed\_request.} \emph{Signed\_request}s are an extended type of OAuth credential that is specific to Facebook.  It enables applications to perform authentication \emph{without} initiating communication with Facebook, unlike the \emph{code} flow.  This is done by signing a data blob containing user information (his or her Facebook user ID, email address, and possibly a valid \emph{code} to obtain \emph{access\_token} upon request) using the \emph{client\_secret} known only to the application owner and Facebook.  To complete the authentication process, the developer needs to verify the signature of \emph{signed\_request} before recognizing the user.

\end{itemize}

We leave out the \emph{refresh\_token} type defined in OAuth 2.0 standard as it is irrelevant to this dissertation.