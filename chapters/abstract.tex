\begin{abstract}

Modern applications integrate third-party services for easier development, additional functionality such as analytics services, and extra revenue including advertising networks.  This integration, however, presents risks to application security and user privacy.  This research addresses integrated applications that incorporate two types of third-party services: (1) services from trusted providers that provide security-critical functionalities to the application such as Single Sign-On (SSO) and file sharing services, and (2) services from untrusted providers that provide other functionalities such as analytics and advertisements.  Unlike traditional library inclusions, integrated applications present new challenges due to the opaqueness of third-party back end service and platform runtime.

For the first type of integration, we assume a benign service provider and our goal is to eliminate misunderstandings between the service provider and the application developer which may lead to security vulnerabilities in the implementation.  We advocate for a systematic approach to discover implicit assumptions and SDK bugs that uses an iterative process to refine system models and uncover needed assumptions.  Our preliminary results for SSO systems have shown significant opportunity and impact --- of the 55 popular applications we've tested, more than half had serious security vulnerabilities due to missing at least one security-critical assumption uncovered by our approach.

To better understand the prevalence of previous discovered vulnerabilities in a larger scale, we present the design and implementation of an automated vulnerability scanner, SSOScan, that can be deployed in an application marketplace or as a stand-alone service.  This testing framework can drive the application automatically and check if a given application is vulnerable by carrying out simulated attacks and monitoring application traffic and behavior.  

Our evaluation results on the top 20,000 websites show that SSOScan is able to automatically check 80\% of the applications for four different types of vulnerabilities, and that approximately 20\% of sites which integrate the SSO service have at least one vulnerability.  

For the second type of integration, the embedding application does not rely on a third-party service for security-critical functionality, but wants to prevent harm to the application and its users from embedded services that may be malicious.  Integrated services often execute as the same principal as the host application code and therefore have full access to application and user data.  To mitigate the potential risks of integrating untrusted code, our approach aims to prevent third-party services from exfiltrating sensitive data or maliciously tampering with the host content.  To this end, we developed two modified browsers that mediates third-party services' access to host web content based on whitelist and blacklist policies.  We also developed automatic policy generators for them, and our experience and evaluation show that the whitelist approach could generate more robust and clear policies than the blacklist approach.  

\end{abstract}