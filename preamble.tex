\usepackage{url}
\usepackage{ifpdf}
% \usepackage{times}
% \usepackage{mathptmx}
% \usepackage{epstopdf}
\usepackage{verbatim}
\usepackage{setspace}
% \usepackage{subfigure}
% \usepackage[noend]{algorithmic}
\usepackage{stmaryrd}
\usepackage{amsmath}
\usepackage{multirow}
\usepackage{listings}
\usepackage{color}
\usepackage[sort]{cite}
\usepackage{relsize}
\usepackage{graphicx}
\usepackage{graphics}
\usepackage{caption}
\usepackage{epsfig}
\usepackage{comment}
\usepackage[nounderscore]{syntax}
%\usepackage{enumitem}
\usepackage{algorithm}
%\usepackage[perpage,para,symbol*]{footmisc}
\usepackage{perpage} %the perpage package
\MakePerPage{footnote} %the perpage package command
\usepackage{algorithmicx}
\usepackage{algpseudocode}
\usepackage{varwidth}
\usepackage{listings}
\usepackage{fixltx2e}
\usepackage{xspace}
\usepackage{makeidx}
\usepackage{threeparttable}
\usepackage{booktabs}
\usepackage[usenames,dvipsnames]{xcolor}

\newcommand\shortsection[1]{\vspace*{6pt}{\noindent\bf #1.}}
\newcommand{\minitab}[2][l]{\begin{tabular}{#1}#2\end{tabular}}

\newcommand{\authnote}[2]{{\textbf{$\mathbf{\big[}$~#1's note:}}
    \textbf{\em\small #2}~{$\mathbf{\big]}$}}

\newcommand{\ynote}[1]{\textcolor{blue}{\bf \emph{Y: #1}}}
\newcommand{\dnote}[1]{\textcolor{red}{\bf \emph{D: #1}}}

%\newcommand{\ynote}[1]{\textcolor{blue}{\bf \emph{}}}
%\newcommand{\dnote}[1]{\textcolor{blue}{\bf \emph{}}}

\newcommand{\TODO}[1]{\textcolor{blue}{TODO:#1}}
\newcommand\keyword[1]{{\sf\small{#1}}}

\algnewcommand{\AND}{\textbf{and}}
\algnewcommand{\NOT}{\textbf{not}}

\newcommand{\code}[1]{{\small\lstinline!#1!}}
\newcommand{\codesmall}[1]{{\scriptsize\lstinline!#1!}}

% from SSOScan
\newcommand{\toolName}{SSOScan\xspace}
% from Oakland15
\newcommand{\vis}{Visualizer\xspace}
\newcommand{\pg}{Policy\-Generator\xspace}
\newcommand{\st}{Script\-Inspector\xspace}
\def\policy{\lstinline[basicstyle=\sffamily\small,keywordstyle={},language=policy]}
\lstdefinelanguage{policy}{
  keywords={sub, root},
  keywordstyle=\bfseries,
  commentstyle=\color{purple}\rmfamily\itseries
}


\newcommand{\var}[1]{{\ttfamily#1}}% variable
\hyphenation{SSO-Scan}
\lstdefinelanguage{JavaScript}{
  keywords={typeof, new, true, false, catch, function, return, null, catch, switch, var, if, in, while, do, else, case, break, eval, window},
  keywordstyle=\color{blue}\bfseries,
  ndkeywords={class, export, boolean, throw, implements, import, this},
  ndkeywordstyle=\color{darkgray}\bfseries,
  identifierstyle=\color{black},
  sensitive=false,
  comment=[l]{//},
  morecomment=[s]{/*}{*/},
  commentstyle=\color{purple}\ttfamily,	
  stringstyle=\color{red}\ttfamily,
  morestring=[b]',
  morestring=[b]"
}

\lstset{language=JavaScript,
    basicstyle=\sffamily\small,
    keywordstyle=\bfseries,
    showstringspaces=false,
		numbers=left,
		numbersep=10pt,
		xleftmargin=.05\textwidth, 
		xrightmargin=.05\textwidth,
		numberstyle=\small\color{blue},
    morekeywords={document, printf}
}